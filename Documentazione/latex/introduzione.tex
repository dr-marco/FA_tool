\section{Introduzione}

Il progetto richiesto e trattato in questo report riguarda il programma disegnato e scritto utile per operazioni e analisi su una rete di automi a stati finiti. 
Obiettivo principale del programma è quello di poter operare su tali automi e generare delle strutture particolari quali dei spazi comportamentali utili per eseguire una successiva analisi e valutazione.

Durante la progettazione e realizzazione del programma sono state adoperate delle scelte utili sia in valutazione della complessità spaziale ma soprattutto temporale del software
sia per usabilità del programma e ipotetico sviluppo futuro. Tali scelte verranno manifestate e argomentate nei capitoli successivi.

La struttura di questo report è suddivisa in pochi capitoli. Subito dopo l'introduzione si parlerà dei strumenti utilizzati per la realizzazione del software, sia in termini di linguaggi di programmazione e di librerie che di tipologie di file dati utilizzati per la creazione ma anche per l'esecuzione del programma. 
Il terzo capitolo affronterà quali sono le strutture dati che sono alla base del progetto e gli algoritmi che operano su tali strutture, commentando brevemente lo pseudocodice delle varie funzioni presenti nel programma; verranno accennate anche alcune delle funzioni di supporto utilizzate nel progetto.
Trattato il programma, si passerà con il quarto capitolo all'analisi delle prestazioni, sia a livello teorico valutando la complessità teorica dell'algoritmo, possibili differenze di complessità con lo pseudocodice originario e le effettive prestazioni del programma in esecuzione, sia in termini di tempo che di memoria.
Infine nel quinto e ultimo capitolo verranno mostrati alcuni esempi di reti di automi che possono essere elaborate dal programma e quali sono i risultati di una loro possibile analisi.

La presente relazione è pensata per essere esaminata assieme alla documentazione dell'elaborato fornito a inizio progetto. Vi potranno essere riferimenti teorici da tale file che non saranno spiegati esaustivamente in questo documento.
Pertanto per qualsiasi precisazione l'invito è quello di esaminare tale documentazione qualora dovessero servire dei chiarimenti.

Il codice del programma e tutta la documentazione è disponibile su \url{https://github.com/dr-marco/FA_tool}